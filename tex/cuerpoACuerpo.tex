Para atacar cuerpo a cuerpo debes estar en la misma casilla que tu objetivo y no estar separado por una barrera. El ataque se realiza usando la habilidad \emph{CLOSE COMBAT}. 

\paragraph{Bloquear}
Cuando un enemigo te ataca en cuerpo a cuerpo, puedes lanzar \emph{CLOSE COMBAT} para bloquear el ataque. Esta se considera una acción rápida. Debes declarar el bloqueo antes de que el enemigo lance sus dados. Cada acierto cancela un acierto del enemigo. Los aciertos extra no tienen efecto.

\paragraph{Carga/Diving Blow}
Cuando primero corres a una casilla y luego llevas a cabo un ataque cuerpo a cuerpo desarmado en el mismo turno recibes +2. Si el ataque pega tu y el enemigo quedáis tumbados. Si fallas solo tu quedas tumbado.

\paragraph{Empujar}
Como acción rápida puedes empujar a otro personaje. Debes realizar un chequeo de \emph{CLOSE COMBAT}. Si tu enemigo tiene mas fuerza que tu necesitas 2 aciertos, si no, necesitas uno. Empujar no causa daño pero puede ser bloqueado.

\paragraph{Desarmar}
Como acción rápida puedes intentar quitar un objeto o arma a alguien que lo tenga en la mano. Lanza \emph{CLOSE COMBAT}. Si el objeto esta sujeto con dos manos requiere dos aciertos si no requiere una. Esta acción puede ser bloqueada.

\paragraph{Agarrar}
Como acción lenta puedes intentar agarrar a un oponente. Lanzas \emph{CLOSE COMBAT} si aciertas tu y el enemigo caéis al suelo y el enemigo queda desarmado. La única acción que alguien agarrado puede realizar es liberarse. Si tienes a alguien agarrado puedes realizar un ataque agarrado, es una acción rápida que no puedes ser bloqueada.

\paragraph{Liberarse}
Cuando estas agarrado puedes liberarte como acción lenta. Lanza \emph{CLOSE COMBAT} enfrentado contra quien te esta agarrando.

\paragraph{Retirarse} \label{page:retreat}
Para moverte cuando tienes un enemigo en la misma casilla que tu debes hacer un chequeo de retirarse, siempre que no haya un barrera entre vosotros. Haz un chequeo de \emph{MOBILITY} si fallas el enemigo tiene un ataque gratuito cuerpo a cuerpo.