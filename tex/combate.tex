\section{Movimiento}
\subfile{movimiento.tex}

\section{Cobertura}
    Existen dos tipos de cobertura parcial y completa. En una casilla con terreno que permite tomar cobertura puede utilizar una acción rápida para \emph{entrar en cobertura}.
    Al entrar en cobertura debes decidir si entras en cobertura parcial o completa.

    \paragraph{Cambiar cobertura}
    Cambiar de cobertura completa a parcial exige una acción gratuita. Cambiar de cobertura parcial a completa es una acción gratuita, pero solo lo puedes hacer en tu turno.

    \paragraph{Dirección de cobertura}
    La cobertura solo tiene efecto si el ataque viene de una angulo de 120 grados, podemos ver un ejemplo en la imagen de la página 59 del manual de juego.

    \paragraph{Misma casilla}
    Para que la cobertura sea valida cuando el ataque viene de la misma casilla, la cobertura deberá ser por una barrera que cruza todo la casilla.

\section{Atacar}

    \subsection{Cuerpo a cuerpo} \label{page:retreat}

    \subsection{Ataque a distancia}

        \subsubsection{Ametralladoras}

    \subsection{Ataque armas pesadas}

        \subsubsection{Tipo de munición}

        \subsubsection{Fuego directo}

        \subsubsection{Fuego indirecto}

        \subsubsection{Granadas y armas arrojadizas}

        \subsubsection{Misiles antitanque}

\section{Daño}

\subsection{Humano y animales}

\subsection{Vehículos}

\section{Estrés}

\section{Explosiones}

\section{Conflictos sociales}

