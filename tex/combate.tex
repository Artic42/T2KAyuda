\section{Movimiento}
    En este juego el movimiento se describe en dos escalas, táctico y estratégico. En combate siempre se moverá en escala táctica, en estos mapas un casilla es una distancia equivalente a 10 metros.

    \subsection{A pie}
        Moverse a pie es una acción rápida, un jugador puede moverse dos casillas sin hacer ningún chequeo si lo hace de pie.
        Levantarse es una acción rápida. Si queremos movernos mas de dos casillas, tras movernos dos casillas, hacemos un chequeo de \emph{MOBILITY}.
        Por cada acierto te puedes mover una casilla más.

        \paragraph{Arrastrarse}
        Si estas tumbado no puedes correr. Pero puedes arrástrate. Arrastrarse funciona igual que moverse pero te mueve una casilla en lugar de dos.
        Con el chequeo de \emph{MOBILITY} solo puedes aumentar tu movimiento en una casilla. El movimiento máximo arrastrándose es 2.

        \paragraph{Mochila}
        Si llevas una mochila tu tirada de movilidad se vera afectada según la tabla \ref{table:1}.

        \begin{table}
            \centering
            \begin{tabular}{ c c }
                \hline
                Mochila & Modificador \\
                \hline
                0 - 1/3 & -1 \\
                1/3 - 2/3 & -2 \\
                2/3 - Lleno & -3 \\
                \hline
            \end{tabular}
            \caption{Modificadores de movilidad por mochila.}
            \label{table:1}
        \end{table}

        \paragraph{Elevación}
        Moverse un terreno hacia terreno más elevado cuesta 2 casillas de movimiento en lugar de 1. Si no tienes dos casillas de movimiento no te puedes mover hacia terreno elevado.

        \paragraph{Ceganales y aguas poco profundas}
        Correr por un cenagal funcionar igual que arrastrase. No se puede arrastrar a traves de cenagales o vados.

        \paragraph{Combate cuerpo a cuerpo}
        Si sabes que hay un enemigo en tu casillas no puedes correr. En su lugar debes retirarte, ver sección \ref{page:retreat}

        \subsubsection{Barreras}

    \subsection{Vehículos}

\section{Cobertura}
    Existen dos tipos de cobertura parcial y completa. En una casilla con terreno que permite tomar cobertura puede utilizar una acción rápida para \emph{entrar en cobertura}.
    Al entrar en cobertura debes decidir si entras en cobertura parcial o completa.

    \paragraph{Cambiar cobertura}
    Cambiar de cobertura completa a parcial exige una acción gratuita. Cambiar de cobertura parcial a completa es una acción gratuita, pero solo lo puedes hacer en tu turno.

    \paragraph{Dirección de cobertura}
    La cobertura solo tiene efecto si el ataque viene de una angulo de 120 grados, podemos ver un ejemplo en la imagen de la página 59 del manual de juego.

    \paragraph{Misma casilla}
    Para que la cobertura sea valida cuando el ataque viene de la misma casilla, la cobertura deberá ser por una barrera que cruza todo la casilla.

\section{Atacar}

    \subsection{Cuerpo a cuerpo} \label{page:retreat}

    \subsection{Ataque a distancia}

        \subsubsection{Ametralladoras}

    \subsection{Ataque armas pesadas}

        \subsubsection{Tipo de munición}

        \subsubsection{Fuego directo}

        \subsubsection{Fuego indirecto}

        \subsubsection{Granadas y armas arrojadizas}

        \subsubsection{Misiles antitanque}

\section{Daño}

\subsection{Humano y animales}

\subsection{Vehículos}

\section{Estrés}

\section{Explosiones}

\section{Conflictos sociales}

