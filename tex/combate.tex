\section{Movimiento}
\subfile{movimiento.tex}

\section{Cobertura}
    Existen dos tipos de cobertura parcial y completa. En una casilla con terreno que permite tomar cobertura puede utilizar una acción rápida para \emph{entrar en cobertura}. Al entrar en cobertura debes decidir si entras en cobertura parcial o completa.

    \paragraph{Cambiar cobertura}
    Cambiar de cobertura completa a parcial exige una acción gratuita. Cambiar de cobertura parcial a completa es una acción gratuita, pero solo lo puedes hacer en tu turno.

    \paragraph{Dirección de cobertura}
    La cobertura solo tiene efecto si el ataque viene de una angulo de 120 grados, podemos ver un ejemplo en la imagen de la página 59 del manual de juego.

    \paragraph{Misma casilla}
    Para que la cobertura sea valida cuando el ataque viene de la misma casilla, la cobertura deberá ser por una barrera que cruza todo la casilla.

\section{Ataque}

    \subsection{Cuerpo a cuerpo(62)}
    \subfile{cuerpoACuerpo.tex}

    \subsection{Ataque a distancia(63)}
    \subfile{ataqueDistancia.tex}       

\section{Armas pesadas}
    La armas pesadas se utilizan con la habilidad armas pesadas y fuerza, pero hay excepciones. Si el arma se encuentra en un trípode o vehículo, se utiliza Agilidad en lugar de Fuerza. Si se trata de artillería, se utiliza inteligencia. Las granadas siguen su propias normas ver \ref{page:granade}.

    \subsection{Tipo de munición(71)}

    Muchas de las armas pesadas se pueden utilizar con diferentes tipos de munición. Estas son las municiones disponibles.

        \paragraph{HE} Alto explosivo normalmente no tienen buena penetración de blindaje. Pero causan una gran explosión.

        \paragraph{HEAT} Alto explosivo antitanque, cuenta con buena penetración de blindaje y explosión.

        \paragraph{AP, APDS, APFSDS} Munición anti blindaje, no causa explosión normalmente.

        \paragraph{WP} Fósforo blanco, arde con un calor intenso, no tienen daño directo solo genera daño de fuego, pagina 79 manual del jugador.

        \paragraph{CHEM} Este tipo de municiones liberan agentes químicos en el objetivo.

        \paragraph{ILUM} Esta munición ilumina la zona objetivo, en un radio igual al dado del tipo de Blast.

    \subsection{Fuego indirecto}
    La artillería puede disparar de manera indirecta. En ese caso una persona estará actuando como observador, guiando los disparos de la artillería. Es necesario que el observador y el personal a cargo de la artillería puedan comunicarse por radio o voz.

    \subsection{Granadas y armas arrojadizas}\label{page:granade}

    \subsection{Misiles antitanque}

\section{Daño}

    \subsection{Humano y animales}

    \subsection{Vehículos}

\section{Estrés}

\section{Explosiones}

\section{Emboscadas}

\section{Overwatch}

\section{Conflictos sociales}

