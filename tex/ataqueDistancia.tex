Realizar un ataque a distancia exige una acción lenta. El ataque se modifica según la tabla de la ayuda rápida o pagina 64 del manual de jugador. El ataque se realiza utilizando la habilidad \emph{RANGED COMBAT}.

\paragraph{Apuntar}
Apuntar exige que tomes una acción lenta. Si no apuntas tendrás un negativo en tu tirada de ataque según aparece en la tabla.

\subsubsection{Alcance}
El alcance descrito en la tabla del arma se considera alcance corto en casillas. El doble de alcance corto se considera alcance medio. El doble de alcance medio se considera alcance largo. El doble de alcance largo se considera extremo. Ejemplo de un arma de alcance 5.

\begin{table}
    \centering
    \begin{tabular}{ c c c }
        \hline
        Alcance & Distancia(Hex) & Modificador \\
        \hline
        Corto & 5 & -0 \\
        Medio & 10 & -1 \\
        Largo & 20 & -2 \\
        Extremo & 40 & -3 \\
        \hline
    \end{tabular}
    \caption{Alcances de arma de rango 5}
    \label{table:2}
\end{table}

\paragraph{Escopetas}
Si el arma que utilizas es una escopeta el modificador de alcance se aplicara al daño en lugar de a la tirada de ataque.

\paragraph{Fuego amigo}
Puedes disparar a un objetivo aunque tu linea de fuego cruce la casilla en la que hay un aliado, si lo haces tu aliado deberá hacer un chequeo CUF. Si fallas, deberás lanzar dos D6 si sacas un acierto el disparo impactará en un aliado al azar en la casilla.

\subsubsection{Ametralladora}
Si estas utilizando una ametralladora, utilizarás \emph{HEAVY WEAPONS} en lugar de \emph{RANGED COMBAT}. Si la ametralladora esta en un trípode o vehículo utilizaras Agilidad en lugar de Fuerza. Si las ametralladoras se dispara sin trípode la tirada de ataque tiene un modificador según la tabla.

    \begin{table}
        \centering
        \begin{tabular}{ c c }
            \hline
            Tipo & Modificador \\
            \hline
            Ligera(LMG) & -2 \\
            Media(MMG) & -3 \\
            Pesada(HMG) & No se puede \\
            \hline
        \end{tabular}
        \caption{Modificado según tipo de ametralladora.}
        \label{table:3}
    \end{table}