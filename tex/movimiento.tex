En este juego el movimiento se describe en dos escalas, táctico y estratégico. En combate siempre se moverá en escala táctica, en estos mapas un casilla es una distancia equivalente a 10 metros.

    \subsection{A pie}
        Moverse a pie es una acción rápida, un jugador puede moverse dos casillas sin hacer ningún chequeo si lo hace de pie.
        Levantarse es una acción rápida. Si queremos movernos mas de dos casillas, tras movernos dos casillas, hacemos un chequeo de \emph{MOBILITY}.
        Por cada acierto te puedes mover una casilla más.

        \paragraph{Arrastrarse}
        Si estas tumbado no puedes correr. Pero puedes arrástrate. Arrastrarse funciona igual que moverse pero te mueve una casilla en lugar de dos.
        Con el chequeo de \emph{MOBILITY} solo puedes aumentar tu movimiento en una casilla. El movimiento máximo arrastrándose es 2.

        \paragraph{Mochila}
        Si llevas una mochila tu tirada de movilidad se vera afectada según la tabla \ref{table:1}.

        \begin{table}
            \centering
            \begin{tabular}{ c c }
                \hline
                Mochila & Modificador \\
                \hline
                0 - 1/3 & -1 \\
                1/3 - 2/3 & -2 \\
                2/3 - Lleno & -3 \\
                \hline
            \end{tabular}
            \caption{Modificadores de movilidad por mochila.}
            \label{table:1}
        \end{table}

        \paragraph{Elevación}
        Moverse un terreno hacia terreno más elevado cuesta 2 casillas de movimiento en lugar de 1. Si no tienes dos casillas de movimiento no te puedes mover hacia terreno elevado.

        \paragraph{Ceganales y aguas poco profundas}
        Correr por un cenagal funcionar igual que arrastrase. No se puede arrastrar a traves de cenagales o vados.

        \paragraph{Combate cuerpo a cuerpo}
        Si sabes que hay un enemigo en tu casillas no puedes correr. En su lugar debes retirarte, ver sección \ref{page:retreat}

        \subsubsection{Barreras}
        Si una barrera cubre una casilla de lado a lado esta barrera bloquea el movimiento en esta casilla.
        Si la barrera es de altura media se puede saltar con un chequeo de \emph{MOBILITY}, esto se considera una acción rapida. Si la barre es alta entonces esta acción consume un acción lenta. Cruzar de barrera solo te mueve a la siguiente casilla si la barrera estaba en la esquina de la casilla.

    \subsection{Vehículo(82)}

    Subirse o bajarse del vehiculo se considera una acción lenta. Si el vehiculo es una motocicleta o bicicleta se convierte en acción rapida.

    El vehiculo se mueve una cantidad de casilla igual a su movimiento utilizando un acción rapida. El movimiento se puede acabar en el angulo que el conductor decida. Para moverse más rapido se puede hacer un chequeo de \emph{DRIVING}, modificado por el terrenoo en el que se ha terminado la primera fase de movimiento. Si hay un acierto el vehiculo volverá a mover su movimiento más una casilla por acierto. Si el chequeo falla el vehiculo se queda "atascado".

        \subsubsection{Atascado(82)}
        Si el vehiculo queda atascado el conductor puede hacer una acción lenta para intentar liberarlo. Hace un chequeo de \emph{DRIVING}.Si el chequeo falla el vehiculo queda "estancado".

        \subsubsection{Estancado(82)}
        Un vehiculo estancado no se puede mover. Para liberarlo se debe salir del vehiculo y hacer un chequeo de \emph{STAMINA}, se recibe +2 si se utiliza otro vehiculo de mismo o mayor peso.

        \subsubsection{Lanzadores de humo(82)}
        Un vehiculo equipado con lanzadores de humo puede activarlo para llenar las seis casillas adyacentes con humo.
